\documentclass[12pt]{article}
\usepackage[latin1]{inputenc}
\author{Aitor Acedo Legarre}
\title{Ejercicios sobre Algoritmos Voraces}
\date{28 de Octubre de 2003} %Forzamos a que ponga esta fecha
\pagestyle{headings}
\begin{document}
\large
\maketitle
\begin{center}{Powered by \LaTeX.}\end{center}
\newpage
\hyphenation{do-mi-na}
\section{Introducci�n del problema}
El problema que tenemos entre manos consiste en dise�ar un algoritmo que deber� encontrar el punto maximal de un conjunto de puntos. Para comprender el concepto de punto marginal de un conjunto de puntos tendremos que tener presente otro concepto como es el de punto dominante, se dice que un punto domina a otro cuando 
la coordenada x del primer punto es mayor o igual que la coordenada x del segundo punto y lo mismo con la segunda coordenada, siendo adem�s puntos diferentes del espacio eucl�deo.
Para modelar el problema hemos utilizado una estructura para representar a un punto, la cual tendr� dos campos que ser�n dos enteros que representar�n las coordenadas del punto.
Para representar el conjunto utilizaremos un vector de dimensi�n fija, cuyo tama�o vendr� determinado por una constante.
El algoritmo para el encontrar el maximal del conjunto de puntos simplemente asignar� el primer punto como posible candidato para ser el maximal e intentar� encontrar un punto de entre los puntos restantes un punto que lo domine, si lo encuentra este nuevo punto pasar� a ser la soluci�n y continuaremos recorriendo el vector hasta que lleguemos al final.

\section{Algoritmo para resolver el problema}
El algoritmo que nos va a resolver el problema tiene la siguiente estructura y est� escrito en una nomenclatura similar a C:
\newpage
\\
\begin{verbatim}
#define TAM 100
struct point{ int x; int y}; 
struct point sol;
struct point C[TAM];

struct point *Maximal(struct point *C){
      int i;
      for(i=1; i<TAM; i++){
              if(C[i].x >= sol.x) && (C[i].y >= sol.y){
                      if(C[i].x != sol.x) && (C[i].y != sol.y) {
                             sol.x = C[i].x;
                             sol.y = C[i].y;
                      }
              }
      }
      return &sol;
}
\end{verbatim}
\end{document}
