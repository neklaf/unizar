\chapter{Evaluaci�n de prestaciones}
%En este cap�tulo tendremos que explicar brevemente el SPT

Las herramientas que realizan este tipo de an�lisis sobre los modelos proporcionan 2 funciones:
\begin{enumerate}
	\item Calcular las prestaciones para una instancia del sistema.
	\item Orientarnos sobre las mejoras del sistema, identificando \emph{cuellos de botella} o recursos cr�ticos.
\end{enumerate}

Aqu� explicaremos un conjunto de conceptos para soportar las ideas centrales del an�lisis de prestaciones.

\section{Conceptos y t�cnicas del an�lisis de prestaciones}

\emph{Scenarios:} caminos de respuesta de nuestro sistema con unos l�mites claramente visibles.
Cada \emph{scenario} es ejecutado por una \emph{clase trabajadora} o una \emph{clase usuario} con una determinada intensidad de carga. A partir de ahora llamaremos a estas clases \emph{workload's}.

\emph{Workload:} son los elementos que definen la ejecuci�n de un \emph{scenario}. Pueden ser de 2 tipos, abiertas y cerradas.
\begin{itemize}
	\item \emph{Open workload:} modelan un flujo de peticiones que llegan con una tasa dada y con unos patrones predeterminados, por ejemplo Poisson.
	\item \emph{Closed workload:} modela un n�mero fijo de potenciales usuarios o trabajos, los cuales realizan un ciclo ejecutando el scenario y consumiendo un retraso externo fuera del sistema.
\end{itemize}

\emph{Scenario steps:} (actividades o pasos del scenario) son los elementos que constituyen un scenario y est�n unidos para formar una secuencia. Es la operaci�n m�s elemental en el nivel de granularidad m�s fino.

\emph{Demandas de recursos:} (para un scenario-step) son sus tambi�n llamadas \emph{host execution demand} y las demandas de todos sus \emph{substeps}.

\emph{Host execution demand:} (para su host device, o dispositivo anfitri�n) es el tiempo de ejecuci�n tomado en su \emph{host device}.

\emph{Recursos:} son modelados como servidores. Los recursos activos son los servidores habituales en modelos de prestaciones y tienen tiempos de servicio.
Los recursos pasivos son asignados y liberados durante la ejecuci�n y tienen tiempos de asignaci�n.

\emph{Tiempo de servicio:} (de un recurso activo) es definido como el \emph{host execution demand} de los pasos que est�n hospedados por el recurso.

Las medidas de prestaciones para un sistema incluyen la utilizaci�n de recursos, tiempos de espera, demandas de ejecuci�n (bien sea en ciclos de CPU o en segundos) y tiempo de respuesta (el actual o el tiempo de reloj para ejecutar un paso del scenario).

\section{Tipos de m�todos de an�lisis de prestacines}

\begin{itemize}
	\item Modelos de colas
	\item Modelos de simulaci�n
	\item Modelos discretos como redes de Petri: definen \emph{tokens} que ejecutan el software, siguiendo la estructura de un scenario detallado.
	Con este m�todo calculamos medidas medias pero puede proporcionar detalladas medidas como los momentos m�s altos y distribuciones.
\end{itemize}