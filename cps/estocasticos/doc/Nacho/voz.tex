\newpage 
% ----------------------------------------------------------  
%%%%%%%%%%%%%%%%%%%%%%%%%%%%%%%%%%%%%%%%%%%%%%%%%%%%%%%%%%%% 
% ----------------------------------------------------------  
\section{Modelos de tr\'afico para comunicaciones de voz}  
 
El modelado de tr\'afico tiene su origen en los sistemas de telefon\'{\i}a  
anal\'ogica convencional.  
En este contexto, los modelos se han basado en procesos estoc\'asticos sencillos 
cuya principal caracter\'{\i}stica es que no contemplan la posibilidad de que  
las trazas de tr\'afico procedentes de las fuentes presenten dependencia a largo 
plazo. 
Es por ello, que se ha definido hist\'oricamente toda una familida de modelos 
{\em cl\'asicos} fundados, siempre y casi exclusivamente, en suposiciones  
markovianas para las tasas de llegada de llamadas (o paquetes de datos) y en  
distribuciones exponenciales para los tiempos de ocupaci\'on de los recursos y 
de servicio de los sistemas. Es decir, siempre se han considerado estos  
modelos de tr\'afico como sistemas sin memoria o, en un sentido m\'as amplio, 
de memoria limitada al conjunto de estados pasados del sistema (habitulamente 
representados mediante una cadena de Markov). 
Seg\'un este \'ultimo planteamiento, la dependencia no se alargar\'{\i}a sino 
que a\~nadir\'{\i}a al modelo cierta memoria, aunque de corto plazo y siempre  
a costa de aumentar considerablemente el n\'umero de estados de la cadena, que  
crecer\'{\i}a exponencialmente con el n\'umero de estados que interese recordar. 
%\vspace{0.35cm} 
 
Alguno de estos modelos cl\'asicos, de hecho, no s\'olo est\'an muy desarrollados, 
sino que est\'an completamente vigentes en muchos de los sistemas y aplicaciones 
m\'as usuales. Los modelos que se han empleado tradicionalmente para caracterizar 
las fuentes de voz aprovechan las caracter\'{\i}sticas del habla humana 
que se presenta a r\'afagas ({\em talk spurts}), con silencios intercalados
entre palabras y entre frases. 
Por ello, lo m\'as habitual es que los codificadores 
de voz (por ejemplo, GSM) incorporen detectores de silencios, durante los cuales no 
transmiten informaci\'on. 
Estos modelos de actividad y silencio presentan comportamientos particulares seg\'un 
el intervalo temporal en el que se analicen. Esto se conoce como el 
fen\'omeno de modelado por niveles, seg\'un el cual el an\'alisis de 
tr\'afico puede realizarse en distintos periodos temporales obteniendo 
resultados  claramente diferentes en funci\'on del intervalo elegido.  
Si se establece una divisi\'on por niveles en funci\'on de estas 
escalas temporales en la que se producen los eventos y la propia naturaleza de los  
mismos, se puede encontrar una clasificaci\'on equivalente a los niveles OSI 
que distingue entre el nivel de celda (N1, relacionado con el flujo de bits  
y referenciado en {\em $\mu$s.}), el nivel de r\'afaga (N2, relacionado  
con la tasa de paquetes y tramas de datos en {\em ms.}), y el nivel de llamada  
(N3, relacionado con la tasa de llegadas en {\em s.}). 
Seg\'un esta divisi\'on, los modelos de actividad y silencio adecuan bastante  
bien sus caracter\'{\i}sticas a los niveles N2 y N3 presentando un cierto tiempo en 
el que transmiten informaci\'on o periodo de actividad (marcado por su  
duraci\'on media $T_{ON}$) seguido de un periodo de silencio (con su  
correspondiente duraci\'on media $T_{OFF}$). As\'{\i} se define el par\'ametro  
de referencia de este modelo como el factor de actividad  
$a=T_{ON}/(T_{ON}+T_{OFF})$. 

A nivel de celda, sin embargo, pueden encontrarse diferencias sustanciales  
entre unas fuentes y otras lo que ha supuesto que aparezcan en la 
literatura diversos modelos como los que se enumeran a continuaci\'on: 
 
\begin{description} 
\item[Modelo ON-OFF.] 
Este primer modelo ampliamente extendido y aceptado, consta de dos estados 
que intentan describir una fuente emisora de informaci\'on a r\'afagas de tal  
manera que en el estado ON se generan paquetes de voz y durante el estado OFF 
se produce silencio. El tiempo de estancia en cada uno de estos estados se  
distribuye exponencialmente con medias $\alpha$ y $\beta$, respectivamente, lo  
que permite expresiones anal\'{\i}ticas cerradas y manejables para establecer 
los par\'ametros b\'asicos del trafico debido a este modelo. 
 
\item[Modelo IPP.] 
Es una variante del modelo anterior que se conoce como modelo de Poisson  
interrumpido (IPP, {\em Interrupt Poisson Processes}), y se caracteriza por ser 
una particularizaci\'on donde el estado activo en vez de presentar transmisi\'on 
continua, se corresponde a llegadas distribuidas exponencialmente con media $\lambda$. 
 
\item[Modelo MMPP.] 
Se corresponde a procesos de Poisson modulados por cadenas de Markov  
(MMPP, {\em Markov Modulated Poisson Processes}),  
tambi\'en llamados procesos doblemente estoc\'asticos por utilizar 
una cadena de Markov (que define la distribuci\'on de probabilidad del  
tr\'afico) como moduladora, y un proceso de Poisson (que genera el tr\'afico 
propiamente dicho) como proceso modulado. 
Este modelo puede aprovechar los estudios y resultados del anterior ya que un  
MMPP de M+1 estados se puede construir de forma evidente a partir de la  
superposici\'on de M procesos IPP independientes e id\'enticamente 
distribuidos. As\'{\i}, mediante los MMPP se puede conseguir que el modelo  
de tr\'afico sea anal\'{\i}ticamente abordable y son especialmente adecuados 
para modelar mezclas de tr\'afico de voz y datos. 
 
\end{description} 
 

Se muestra a continuaci\'on un esquema de los diagramas de Markov correspondientes
a cada uno de los modelos de voz presentados.

%%% fichero DOC adjunto %%%



